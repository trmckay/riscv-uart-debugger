\documentclass[10pt,a4paper]{article}
\usepackage[utf8]{inputenc}
\usepackage{amsmath}
\usepackage{amsfonts}
\usepackage{amssymb}
\usepackage{graphicx}
\usepackage{float}
\usepackage[left=1in,right=1in,top=0.5in,bottom=1in]{geometry}

\title{Pipeline control protocol}
\date{Summer 2020}

\graphicspath{{./figures/}}

\begin{document}

\maketitle

\section{Overview}

This protocol aims to make hardware add-ons for the Otter easier to develop. Control flow operations
allow for easy manipulation of the state of the pipeline as well as many low-level operations such
as memory and register editing, pause, or resume. The module is designed for use with a debugger,
but a standard method for controlling the pipeline can be useful in many other scenarios. Any
external application or module that consists of these basic operations can be more easily
implemented with this module, including ones that interface with I/O.

\section{Connections}

\begin{figure}[H]
	\includegraphics[width=\textwidth]{pipeline_db}
\end{figure}
\medskip

The module connects to the MCU, register file, and memory. Commands are recieved via serial and
decoded within the module. Both \emph{srx} and \emph{stx} can be ignored initially. The signals \emph{rf\_wr, rf\_rd, addr,} and
\emph{d\_in} connect to the register file for reads and writes. The \emph{mem\_wr}, \emph{mem\_rd},
\emph{addr}, and \emph{d\_in} connect to the memory or memory hub for reads and writes. Finally,
\emph{resume, flush}, and \emph{reset} control the flow of the pipeline. The \emph{valid} signal
determines wether or not the controller is issuing a command.

\section{Protocol}

It can be assumed that reads, writes, and resumes will only be issued while the MCU is paused. It
can also be assumed that only one command will occur at any time. If any two commands are issued at the
same time, it may be helpful to set \emph{error} and perhaps even do some error handling.

\begin{enumerate}

    \item\textbf{ready}\\
    Indicates that the debugger is issuing a command. The MCU should respond accordingly on the next positive clock edge.
    The control signals issued by the controller will hold until the \emph{mcu\_busy} goes low.

    \item\textbf{resume}\\
    The MCU should resume normal operation.

    \newpage
    \item\textbf{pause}\\
    The MCU should appear stop execution after the current instruction completes. The
    \emph{mcu\_busy} signal should be high from the next positive edge until the pipeline is
    completely paused. This should be implemented by pausing the PC so that it does not increment
    on the next positive edge and squashing any susequent instructions. Additionally, the pipeline
    must be flushed such that all instructions in the pipeline when the \emph{pause} was
    issued are completed. When this is done, \emph{mcu\_busy} should go low. The MCU is now in a
    state where external modules can interface with the processor without interfering with
    program execution.

    \item\textbf{reset}\\
    The program counter should be set to zero. It is worth noting that this only the position of
    execution, not the contents of memory. Clearing can be optionally implemented by
    pausing the MCU while writing zeros to the entire memory, only resetting the PC when done.

    \item\textbf{mem\_rd}\\
    The MCU should read the memory at \emph{addr}. The \emph{mcu\_busy} should go
     high at or before the next positive clock edge until the memory read is complete. The
    controller will attempt to read the data \emph{d\_rd} on the first positive clock edge that
    \emph{mcu\_busy} is low. If the address is out of range, set \emph{error} to high.
    
    \item\textbf{mem\_wr}\\
    The MCU should write the data \emph{d\_in} to the memory at \emph{addr}. The \emph{mcu\_busy} should go
    high at or before the next positive clock edge until the memory write is complete.
    If the address is out of range, do nothing.

    \item\textbf{rf\_rd}\\
    The MCU should read the register file at \emph{addr}. The \emph{mcu\_busy} should go
    high at or before the next positive clock edge until the read is complete. However, it is
    likely that the register file supports asynchronous reads and that \emph{mcu\_busy} never
    needs to go high. The controller will attempt to read the data \emph{d\_rd} on the first positive clock edge that
    \emph{mcu\_busy} is low. In the case of the register file, this will likely be very the next
    cycle. If the address is out of range, set \emph{error} to high.


    \item\textbf{rf\_wr}\\
    The MCU should write the data \emph{d\_in} to the memory at \emph{addr}. The \emph{mcu\_busy} should go
    high at or before the next positive clock edge until the memory write is complete. This should
    only take one cycle for the register file; if so, \emph{mcu\_busy} does not need to be high.
    If the address is out of range, set \emph{error} to high.

\end{enumerate}

\vspace*{\fill}
\begin{center}
    \noindent Contact Trevor McKay with questions.\\
    trs.mckay@gmail.com
\end{center}

\end{document}
